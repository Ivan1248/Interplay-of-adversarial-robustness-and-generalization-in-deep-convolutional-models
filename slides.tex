\documentclass{beamer}
\usefonttheme{professionalfonts}  % serif math
\setbeamertemplate{frametitle continuation}{} %{(\insertcontinuationcount)}

\iffalse
\usepackage{pgfpages}
\setbeameroption{show notes}
\setbeameroption{show notes on second screen=right}
% pdfpc slajdovi.pdf --notes=right
\fi

\usepackage[scaled]{beramono}				% sans-serif monospace
\input{imports/font}
%\input{imports/text}
%%text%%
\iftrue
	\raggedright						% bez desnog poravnavanja
	\raggedbottom
	\usepackage{caption}
	\captionsetup{%
		justification=raggedright,
	}
	\usepackage{etoolbox}
	\makeatletter
	\patchcmd{\@dottedtocline}
	{\rightskip\@tocrmarg}
	{\rightskip\@tocrmarg plus 4em \hyphenpenalty\@M}
	{}{}
	\makeatother
	\setlength{\parindent}{1em}	 % uvlačenje ulomaka
	\usepackage{indentfirst}	 % uvlačenje prvog ulomka
	\setlength{\parskip}{0.5em}	 % razmak između ulomaka
	
	\usepackage[multiple, bottom]{footmisc}	 % višestruke fusnote, poslije slika/tablica
	
	\renewcommand*{\UrlFont}{\footnotesize}
	
	% colors	
	\iffalse
	\usepackage{xcolor}
	\usepackage{color}
	\hypersetup{
		colorlinks,
		linkcolor={blue!60!green!50!black},  % xcolor package
		citecolor={green!40!black},
		urlcolor={blue!75!green!30!black}
	}
	\definecolor{bluekeywords}{rgb}{0.13,0.13,1}  % color package
	\definecolor{greencomments}{rgb}{0,0.5,0}
	\definecolor{redstrings}{rgb}{0.9,0,0}
	\fi
	
	\DeclareTextFontCommand{\textbf}{\bfseries}
\fi
%%%%%%%%
\input{imports/math}
\input{imports/tables}
\input{imports/figures}

%\usepackage{natbib}
\usepackage[numbers]{natbib}
\renewcommand*{\bibfont}{\scriptsize}

\iftrue
\usepackage[croatian]{babel}
\usepackage[utf8x]{inputenc}	
\fi

\mode<presentation>
{
	\usetheme{Boadilla}      % or try Darmstadt, Madrid, Warsaw, ...
	\usecolortheme{orchid} % or try albatross, beaver, crane, ...
	\usefonttheme{structurebold}  % or try default, serif, structurebold, ...
	\setbeamertemplate{navigation symbols}{}
	\setbeamertemplate{caption}[numbered]
}
\setbeamercolor{structure}{fg=blue!75!green!80!black}	

%\setbeamertemplate{itemize items}[default]
%\setbeamertemplate{enumerate items}[default]
\setbeamertemplate{section in toc}[circle]
\setbeamertemplate{subsection in toc}[circle]
\setbeamertemplate{items}[circle]
\setbeamertemplate{blocks}[default]
\setbeamertemplate{footline}
{
	\leavevmode%
	\hbox{%
		\begin{beamercolorbox}[wd=.15\paperwidth,ht=2.25ex,dp=1ex,center]{author in head/foot}%
			%\usebeamerfont{author in head/foot}\insertshortauthor
		\end{beamercolorbox}%
		\begin{beamercolorbox}[wd=.7\paperwidth,ht=2.25ex,dp=1ex,center]{author in head/foot}%
			\usebeamerfont{title in head/foot}\insertshorttitle  %\hspace*{3em}
		\end{beamercolorbox}%
		\begin{beamercolorbox}[wd=.15\paperwidth,ht=2.25ex,dp=1ex,right]{author in head/foot}%
			\insertframenumber{} / \inserttotalframenumber\hspace*{1ex}
		\end{beamercolorbox}
	}%
	\vskip0pt%
}

\iftrue
\AtBeginSection[]
{
	\begin{frame}<beamer>
		\frametitle{Content}
		\tableofcontents[currentsection]
	\end{frame}
}
\fi


\title{Interplay of adversarial robustness and generalization in deep convolutional models}
\author{Ivan Grubišić \newline \emph{Mentor:} Siniša Šegvić}
\institute{Faculty of Electrical Engineering and Computing}
\date{}


\begin{document}
	
\begin{frame}
  \titlepage
\end{frame}

\begin{frame}{Content}
  \tableofcontents
\end{frame}


\section{Adversarial example definitions}

\begin{frame}{Non-robustness of machine learning algorithms}
	\begin{itemize}
		\item The performance of current state-of-the-art machine learning algorithms is \textbf{highly sensitive to input corruption, domain-shift, out-of-distribution inputs and inputs crafted to fool them} and they often make \textbf{overconfident predictions} \citep{Hendrycks:2016:BDMOODE,Ganin:2015:UDAB,Nguyen:2015:DNNEFHCPUI,Hendrycks:2019:BNNRCCP,Engstrom:2017:RTSFCST,Szegedy:2013:IPNN}.
		\item Perhaps most surprisingly, an input (e.g. image) can be slightly (even imperceptibly) modified to cause a misprediction.
		\begin{itemize}
		    \item Performing a single small gradient descent step on an image in a direction of increasing the loss is often enough to fool a classifier \citep{Goodfellow:2014:EHAE}. 	
		    \item Often, there even exists a perturbation that modifies a single pixel and causes misclassification \cite{Su:2017:OPAFDNN}. 	
	    \end{itemize}
	\end{itemize}
	\begin{figure}[htbp!]
	\centering
	{\small
		\begin{tabular}{>{\centering\arraybackslash}m{.22\columnwidth}m{.1in}>{\centering\arraybackslash}m{.22\columnwidth}m{.05in}>{\centering\arraybackslash}m{.22\columnwidth}}
			\centering\arraybackslash
			%abs max for panda was 138, eps was 1., so relative eps is ~.007
			\includegraphics[width=.22\columnwidth]{adversarial-examples/panda_577.png} &%
			\centering\arraybackslash%
			$\centering +\ \epsilon \cdot$ &%
			\includegraphics[width=.22\columnwidth]{adversarial-examples/nematode_082.png} &%
			$\centering =$ & %
			\includegraphics[width=.22\columnwidth]{adversarial-examples/gibbon_993.png} \\
			$\centering \vec x$     &%
			& $\sgn\del{\nabla_{\vec x} L(y,h(\vec x))}$ & & $\tilde{\vec x}$ \\
			\emph{panda} (0.577) & & & & \emph{gibbon} (0.993) 
		\end{tabular}
	}
	\caption{Generation of an adversarial example with FGSM, a single step attack. Italic words and numbers represent classes and confidences. The images are from \citet{Goodfellow:2014:EHAE}.}
	\label{fig:fgsm-adversarial-example}
\end{figure}
\end{frame}


\section{Adversarial examples}

\begin{frame}{Adversarial examples}
\begin{itemize}
	\item The existence of adversarial examples indicates that current algorithms perform well without actually \textbf{understanding data} (in a way similar to humans).
	\item 
\end{itemize}
\end{frame}

\begin{frame}{Properties of adversarial examples}
\begin{itemize}
    \item In the vicinity of an input, there are usually 
\end{itemize}
\end{frame}

\note[itemize]{
	\item\item 0:20 / 10:20
}

\end{document}
